\documentclass[12pt]{scrartcl}
\usepackage[sexy]{evan}
\usepackage{graphicx}

\usepackage{answers}
\Newassociation{hint}{hintitem}{all-hints}
\renewcommand{\solutionextension}{out}
\renewenvironment{hintitem}[1]{\item[\bfseries #1.]}{}
\declaretheorem[style=thmbluebox,name={Theorem}]{thm}

 %Sets
\newcommand{\N}{\mathbb{N}}
\newcommand{\Z}{\mathbb{Z}}
\newcommand{\F}{\mathbb{F}}
\newcommand{\Q}{\mathbb{Q}}
\newcommand{\R}{\mathbb{R}}
\newcommand{\C}{\mathbb C}
\newcommand{\T}{\mathbb T}
\renewcommand{\hat}{\widehat}
\renewcommand{\tilde}{\widetilde}

\renewcommand{\hat}{\widehat}
\newcommand{\<}{\langle}
\renewcommand{\>}{\rangle}
\newcommand\at[2]{\left.#1\right|_{#2}}
\newcommand\p[1]{\frac{\partial}{\partial#1}}
\newcommand\pd[2]{\frac{\partial#1}{\partial#2}}
 
\let \phi \varphi
\let \mc \mathcal
\let \ol \overline
\let \eps \varepsilon
\let \tensor \otimes
%From Topology
\newcommand{\cT}{\mathcal{T}}
\newcommand{\cB}{\mathcal{B}}
\newcommand{\cC}{\mathcal{C}}
\newcommand{\cH}{\mathcal{H}}

\newcommand{\supp}{\text{supp }}


\newcommand{\aint}{\mathrel{\int\!\!\!\!\!\!-}}
\let \grad \nabla

\begin{document}
\title{Tensors}
\author{Vishal Raman}
\maketitle
\begin{abstract}
These notes correspond to Chapter 12 of Lee, \textit{Smooth Manifolds} on Tensors.  We define multilinear maps in order to construct tensors and tensors fields on manifolds.  We also introduce symmetric and alternating tensors, as well as tensor fields and bundles on smooth manifolds.  
\end{abstract}
\tableofcontents
\pagebreak
\section{Multilinear Algebra and the Tensor Product}
\begin{definition} Suppose $V_1, \dots, V_k$ and $W$ are vector spaces.  A map $F: V_1 \times \dots \times V_k \to W$ is \textbf{multilinear} if it is linear as a function of each variable separately when the others are held fixed.
\end{definition}
Some common examples include:
\begin{itemize}
\item The dot product
\item The cross product
\item The determinant
\item The bracket in a Lie algebra
\end{itemize}
\begin{example}[Tensor Products of Covectors] Suppose $V$ is a vector space, and $\omega, \eta \in V^*$.  Define $\omega \tensor \eta: V \times V \to R$ by 
$$\omega \tensor \eta(v_1, v_2) = \omega(v_1) \eta(v_2).$$
The linearity of $\omega$ and $\eta$ implies that $\omega \tensor \eta$ is a bilinear function of $v_1$ and $v_2$.  
\end{example}
\begin{definition} Given $V_1, \dots, V_k, W_1, \dots, W_\ell$ real vector spaces and functions $F \in L(V_1, \dots, V_k; \R)$, $G \in L(W_1, \dots, W_\ell; \R)$, define the \textbf{tensor product} $F \tensor G$ by
$$F \tensor G(v_1, \dots, v_k, w_1, \dots, w_\ell) = F(v_1, \dots, v_k)G(w_1, \dots, w_\ell).$$
\end{definition}
\begin{proposition} Given $V_1, \dots, V_k$ real vector spaces of dimensions $n_1, \dots, n_k$, if $(E_1^{(j)}, \dots, E_{n_j}^{(j)})$ is a basis for $V_j$ with corresponding dual basis $(\epsilon_{(j)}^1, \dots, \epsilon_{(j)}^{n_j})$, the set 
$$\mc B = \{e_{(1)}^{i_1} \tensor \dots \tensor \epsilon_{(k)}^{i_k}: 1 \le i_1 \le n_1, \dots, 1 \le i_k \le n_k\}$$
is a basis for $L(V_1, \dots, V_k; \R)$, which has dimension equal to $n_1 \dots n_k$.
\end{proposition}
\section{Tensors on Vector Spaces}
\begin{definition}Given a vector space $V$, $\dim V < \infty$, define 
$T^k(V^*) = V^* \tensor \dots \tensor V^*$ to be the space of multilinear maps on $V \times \dots \times V \to \R$.
\end{definition}
\begin{definition} We can define a mixed tensor $T^{(k, \ell)}(V) = V \tensor \dots \tensor V \tensor V^* \tensor \dots \tensor V^*$.
\end{definition}
\section{Symmetric and Alternating Tensors}
\begin{definition} Let $V$ be a finite dimensional space.  $\alpha \in T^k(V^*)$ is said to be symmetric is if it invariant under interchanging pairs of elements.  The set of symmetric covariant $k$-tensors is a linear subspace denoted by $\Sigma^k(V^*)$.
\end{definition}
We can construct $\Sigma^k(V^*)$ as the projection from $T^k(V^*)$ under the symmetric group given by 
$$\operatorname{Sym}{\alpha} = \frac{1}{k!} \sum_{\sigma \in S_k} {}^\sigma \alpha.$$
More explicitly,
$$(\operatorname{Sym}{\alpha})(v_1, \dots, v_k) = \frac{1}{k!} \sum_{\sigma \in S_k} \alpha(v_{\sigma(1)}, \dots, v_{\sigma(k)}).$$

Note that $\operatorname{Sym}{\alpha}$ is symmetric and it is the identity if and only if $\alpha$ is symmetric.  

\begin{definition} If $\alpha$ and $\beta$ are symmetric tensors on $V$, we define $\alpha \beta = \operatorname{Sym}(\alpha \tensor \beta)$, the symmetric product.  
\end{definition}
 
 \begin{definition} $\alpha \in T^k(V^*)$ is said to be alternating if it is skew-symmetric.  These are also called exterior forms.
 \end{definition}
 \section{Tensors and Tensor Fields on Manifolds}
 
 \end{document}
