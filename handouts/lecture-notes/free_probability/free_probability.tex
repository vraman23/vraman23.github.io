\documentclass[11pt]{scrartcl}
\usepackage[sexy]{evan}
\usepackage{graphicx}
\usepackage{mathtools}
\usepackage{listings}
\usepackage{quiver}

 %Sets
\newcommand{\N}{\mathbb{N}}
\newcommand{\Z}{\mathbb{Z}}
\newcommand{\F}{\mathbb{F}}
\newcommand{\Q}{\mathbb{Q}}
\newcommand{\R}{\mathbb{R}}
\newcommand{\C}{\mathbb C}
\newcommand{\T}{\mathbb T}
\newcommand{\PP}{\mathbb P}
\newcommand{\supp}{\text{supp }}
\newcommand{\E}{\mathbb E}
\newcommand{\cov}{\operatorname{cov}}
\renewcommand{\Re}{\operatorname{Re}}

\DeclareMathOperator*{\Var}{Var}
\newcommand{\Hol}{\operatorname{Hol}}

\let \hat \widehat
\let \mc \mathcal
\let \p \partial
\let \bar \overline
\let \eps \varepsilon
\newcommand\at[2]{\left.#1\right|_{#2}}

%From Topology
\newcommand{\cT}{\mathcal{T}}
\newcommand{\cB}{\mathcal{B}}
\newcommand{\cC}{\mathcal{C}}
\newcommand{\cH}{\mathcal{H}}

%Indicators 
\newcommand{\1}{\textbf{1}} % vector of all 1's
\newcommand{\I}[1]{\mathbb{I}{\left\{#1\right\}}} % indicator function


\usepackage{answers}
\Newassociation{hint}{hintitem}{all-hints}
\renewcommand{\solutionextension}{out}
\renewenvironment{hintitem}[1]{\item[\bfseries #1.]}{}
\declaretheorem[style=thmbluebox,name={Problem}, numberwithin=section]{prob}

\begin{document}
\title{Math 212}
\author{Vishal Raman}
\thispagestyle{empty}
$ $
\vfill
\begin{center}

\centerline{\huge \textbf{Notes on Free Probability}}
\centerline{Advisor: Dan-Virgil Voiculescu, Fall 2021}
\centerline{Vishal Raman}
\end{center}
\vfill
$ $
\newpage
\thispagestyle{empty}
\tableofcontents
\newpage
%\maketitle

\section{Free Products}
We first define the notion of a free product for objects.  They can all be boiled down to universal products in the categorical sense.  

\begin{definition}[Free Product of Groups] let $(G_i)_{i \in I}$ denote a family of groups.  The \textit{group free product}, denoted $*_{i \in I} G_i$ is the unique group $G$ up to isomorphsim with homomorphisms $\psi_{i} : G_i \to G$ so that given any group $H$ and $\phi_i: G_i \to H$, there exists a unique homomorphism $\Phi =  *_{i \in I} \phi_i: G \to H$ making the diagram commute:
\[\begin{tikzcd}
	{G_i} && G \\
	H
	\arrow["{\psi_i}", from=1-1, to=1-3]
	\arrow["{\phi_i}"', from=1-1, to=2-1]
	\arrow["\Phi", from=1-3, to=2-1]
\end{tikzcd}\]
\end{definition}
The free product can also be constructed as 
$$H = \{g_1g_2\dots g_n : g_j \in G_{i_j} \setminus \{e\}, i_1 \ne i_2 \ne \dots \ne i_n\} \cup \{\emptyset\},$$
with multiplication defined by concatenation followed by reduction.  
\begin{definition}[Free Product of Groups] let $(A_i)_{i \in I}$ denote a family of unital algebras.  The \textit{unital algebra free product}, denoted $*_{i \in I} A_i$ is the unique unital algebra $A$ up to isomorphsim with homomorphisms $\psi_{i} : A_i \to A$ so that given any group $B$ and $\phi_i: A_i \to B$, there exists a unique homomorphism $\Phi: *_{i \in I} A \to B$ making the diagram commute:
\[\begin{tikzcd}
	{A_i} && A \\
	B
	\arrow["{\psi_i}", from=1-1, to=1-3]
	\arrow["{\phi_i}"', from=1-1, to=2-1]
	\arrow["\Phi", from=1-3, to=2-1]
\end{tikzcd}\]
\end{definition}
As a vector space, we construct the unital algebra free product by first taking a basis formed by 
$$B = \{a_1a_2\dots a_n : a_j \in A_{i_j} : i_1 \ne i_2 \ne \dots i_n\},$$
then taking the quotient with the subspace generated by relations
$$a_1 \dots a_{j_1}(\lambda a_j^{(0)} + \mu a_j^{(1)})a_{j+1} \dots a_n, a_j = 1.$$
We also have the following proposition:
\begin{proposition} If $A_i = \C1 \oplus V_i$, then
$$*_{i \in I} A_i \cong \C1 \oplus \bigoplus_{n \ge 1} \left (\bigoplus_{i_1\ne i_2 \ne \dots \ne i_n} V_{i_1} \otimes \dots \otimes V_{i_n}\right).$$
\end{proposition}






References:
[1] Free Random Variables, D. V. Voiculescu, J. H. Dykema, A. Nica
\end{document}