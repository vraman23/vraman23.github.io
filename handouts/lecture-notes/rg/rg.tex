\documentclass[11pt]{scrartcl}
\usepackage[sexy]{evan}
\usepackage{graphicx}
\usepackage{mathtools}
\usepackage{listings}

 %Sets
\newcommand{\N}{\mathbb{N}}
\newcommand{\Z}{\mathbb{Z}}
\newcommand{\F}{\mathbb{F}}
\newcommand{\Q}{\mathbb{Q}}
\newcommand{\R}{\mathbb{R}}
\newcommand{\C}{\mathbb C}
\newcommand{\T}{\mathbb T}
\newcommand{\PP}{\mathbb P}
\newcommand{\supp}{\text{supp }}
\newcommand{\E}{\mathbb E}
\newcommand{\cov}{\operatorname{cov}}
\renewcommand{\Re}{\operatorname{Re}}
\newcommand{\<}{\langle}
\renewcommand{\>}{\rangle}


\DeclareMathOperator*{\Var}{Var}
\newcommand{\Hol}{\operatorname{Hol}}

\let \phi \varphi
\let \tensor \otimes
\let \hat \widehat
\let \mc \mathcal
\let \mb \mathbb
\let \p \partial
\let \bar \overline
\let \eps \varepsilon
\newcommand\at[2]{\left.#1\right|_{#2}}

%From Topology
\newcommand{\cT}{\mathcal{T}}
\newcommand{\cB}{\mathcal{B}}
\newcommand{\cC}{\mathcal{C}}
\newcommand{\cH}{\mathcal{H}}

%Indicators 
\newcommand{\1}{\textbf{1}} % vector of all 1's
\newcommand{\I}[1]{\mathbb{I}{\left\{#1\right\}}} % indicator function


\usepackage{answers}
\Newassociation{hint}{hintitem}{all-hints}
\renewcommand{\solutionextension}{out}
\renewenvironment{hintitem}[1]{\item[\bfseries #1.]}{}
\declaretheorem[style=thmbluebox,name={Problem}, numberwithin=section]{prob}

\begin{document}
\title{RGOM}
\author{Vishal Raman}

\begin{center}
\centerline{ }
\centerline{\huge \textbf{Geometry}}
\centerline{written by Vishal Raman}
\end{center}
\begin{abstract}
We present detailed expository notes on topics in manifold theory, with plans to cover Riemannian Manifolds, Symplectic Manifolds, and Complex Manifolds. We assume familiarity with smooth manifolds(Lee, Smooth Manifoldsd) Any typos or mistakes are my own - please redirect them to \href{vraman@berkeley.edu}{my email}. 
\end{abstract}
\tableofcontents
%\maketitle
\pagebreak
$ $
\vfill
\centerline{\huge \textbf{Part I}}
\centerline{}
\centerline{\huge \textbf{Riemannian Geometry}}
\vfill
$ $
\section{Review of Smooth Manifolds}
\subsection{Tensors on a Vector Space}
Let $V$ be a finite-dimensional vector space.  Recall the dual space $V^*$, the set of covectors on $V$.  We denote the natural pairing $V^* \times V \to \R$ by the notation $(\omega, X) \mapsto \omega(X)$ for $\omega \in V^*, X \in V$.

\begin{definition}[Covariant Tensor] A covariant $k$-tensor on $V$ is a multilinear map $$F: \underbrace{V \times \dots \times V}_{k \text{ times}}\to \R.$$
The space of covariant $k$-tensors on $V$ is denoted $T^k(V)$.
\end{definition}

\begin{definition}[Contravariant Tensor] A contravariant $k$-tensor on $V$ is a multilinear map $$F: \underbrace{V^* \times \dots \times V^*}_{k \text{ times}} \to \R.$$
The space of contravariant $k$-tensors on $V$ is denoted $T_k(V)$.
\end{definition}

\begin{definition}[Mixed Tensor] A mixed $\binom{k}{l}$-tensor on $V$ is a multilinear map $$F: \underbrace{V^* \times \dots \times V^*}_{l \text{ times}} \underbrace{\times V \times \dots \times V}_{k \text{ times}} \to \R.$$
The space of mixed $\binom{k}{l}$-tensors on $V$ is denoted $T^k_l(V)$.
\end{definition}

Some identifications:
\begin{itemize}
\item $T_0^k(V) = T^k(V)$, $T^k_0(V) = T^k_0(V),$.
\item $T^1(V) = V^*$, $T_1(V) = V^{**} = V$.
\item $T^0(V) = \R$.
\item $T^1_1(V) = \End(V)$, the space of linear endomorphisms of $V$.
\end{itemize}

The last identification is a consequence of the following lemma:
\begin{lemma} Let $V$ be a finite-dimensional vector space.  There is a natural isomorphism between $T^k_{l + 1}(V)$ and the space of multilinear maps
$$\underbrace{V^* \times \dots \times V^*}_{l} \times \underbrace{V \times \dots \times V}_k \to V.$$
\end{lemma}
\begin{definition}[Tensor Product] If $F \in T_l^k(V)$ and $G \in T_q^p(V)$, the tensor $F \tensor G \in T_{l + q}^{k + p}(V)$ is defined by 
$$F \tensor G(\omega^1, \dots, \omega^{l + q}, X_1, \dots, X_{k +p}) = F(\omega^1, \dots, \omega^l, X_1, \dots, X_k)G(\omega^{l + 1}, \dots, \omega^{l + q}, X_{k + 1}, \dots, X_{k + p}).$$
\end{definition}

If $(E_1, \dots, E_n)$ is a basis for $V$ and $(\phi^1, \dots, \phi^n)$ denotes the corresponding dual basis for $V^*$(defined by $\phi^i(E_j) = \delta_{ij}$), a basis for $T_l^k(V)$ is given by the set of tensors of the form
$$E_{j_1} \tensor \dots \tensor E_{j_l} \tensor \phi^{i_1} \tensor \dots \tensor \phi^{i_k}.$$


\newcommand{\tr}{\operatorname{tr}}

\begin{definition}[Trace]
Using Lemma $1.4$, we can define the \textit{trace operator} given by $\tr: T_{l+1}^{k + 1}(V) \to T^{k}_l(V)$ where $(\tr F)(\omega^1, \dots, \omega^k, v_1, \dots, v_l)$ is the trace of the tensor
$$F(\omega^1, \dots, \omega^k, \cdot, v_1, \dots, v_l, \cdot) \in T_1^1(V).$$
\end{definition}

\subsection{Vector Bundles and Vector Fields}
\subsection{Tensor Fields}
\subsection{Lie Theory}

\section{Riemannian Metrics}
\begin{definition}[Riemannian Metric] Let $M$ be a smooth manifold.  A \textit{Riemannian metric} on $M$ is a smooth covariant $2$-tensor field $g \in \mc T^2(M)$ whose value $g_p$ at each $p \in M$ is an inner product on $T_pM$; i. e., $g$ is a symmetric $2$-tensor field that is positive definite in the sense that $g_p(v, v) \ge 0$ for each $p \in M$ and each $v \in T_pM$, with equality if and only if $v = 0$.
\end{definition}

\begin{definition}[Riemannian Manifold] A \textit{Riemannian manifold} is a pair $(M, g)$ where $M$ is a smooth manifold and $g$ is a Riemannian metric on $M$.
\end{definition}

\begin{proposition} Every smooth manifold admits a Riemannian metric.
\end{proposition}
\begin{proof}
Let $M^n$ be a smooth manifold with a corresponding covering of smooth charts $(U_\alpha, \phi_\alpha)$.  In each of the coordinate domains, there is a Riemannian metric $g_{\alpha} = \phi_{\alpha}^* \bar{g}$, where $\bar{g} = \delta_{ij} dx^i dx^j$ is the Euclidean metric on $\R^n$.  Now, if we choose $\{\psi_{\alpha}\}$ to be a smooth partition of unity subordinate to $\{U_{\alpha}\}$, then, we can define $g = \sum_{\alpha} \phi_{\alpha} g_{\alpha}$, where each term is interpreted to be zero outside the support of $\phi_{\alpha}$.

By local finiteness, there are only finitely many terms in a neighborhood of each point, so this defines a smooth tensor field.  It is also symmetric by construction.  Finally, if $v \in T_pM$ is nonzero,
$$g_p(v, v) = \sum_{\alpha} \psi_{\alpha}(p) g_{\alpha}\vert_p(v, v) > 0$$
since $g_{\alpha}\vert_p(v, v) > 0$ and at least one of the $\psi_{\alpha}(p) > 0$.
\end{proof}

We can similarly define a Riemannian manifold with boundary when $M$ is a smooth manifold with boundary.  

We will use the notation $\<v, w\>_g = g_p(v, w)$ since $g_p$ is an inner product on $T_pM$.  This motivates the notion of angles, lengths, and orthogonality.  
\subsection{Isometries}
Suppose $(M, g)$ and $(\tilde{M}, \tilde{g})$ are Riemannian manifolds.  
\begin{definition}[Isometry] An \textit{isometry} from $(M, g)$ to $(\tilde{M}, \tilde{g})$ is a diffeomorphism $\phi: M \to \tilde{M}$ such that $\phi^* \tilde{g}  = g$.  This is equivalent to the requirement that $\phi$ is a bijection and $d\phi_p: T_pM \to T_{\phi(p)}\tilde{M}$ is a linear isometry.  
\end{definition}
We denote the $\operatorname{Iso}(M, g)$ as the isometry group of $(M, g)$ under composition.  
\begin{definition}[Local Isometry] A map $\phi: M \to \tilde{M}$ is a local isometry if each point $p \in M$ has a neighborhood $U$ such that $\phi\vert_U$ is an isometry onto an open subset of $\tilde{M}$.
\end{definition}
\begin{exercise} Prove that if $(M, g)$ and $(\tilde M, \tilde g)$ are Riemannian manifolds of the same dimension, a smooth map $\phi: M \to \tilde{M}$ is a local isometry if and only if $\phi^* \tilde{g} = g$.
\end{exercise}
\begin{proof}
The forward direction is obvious.  If $\phi$ is a local isometry, then $\phi\vert_U ^*\tilde{g} = g$ for all $p \in M$ and some open $U \ni p$, and it follows for all of $M$ (consider the definition in terms of the corresponding tangent spaces).

For the reverse direction, take $p \in M$. Note that if $d\phi_p(v) = 0$ for $v \in T_pM$, then, we have 
$$\tilde{g}(d\phi_p(v), d\phi_p(v)) = g(v, v) = 0,$$
which implies that $v = 0$.  Therefore $d\phi_p$ is invertible, so it follows from the inverse function theorem that there exists $U \ni p$ such that $\phi\vert_U$ is a diffeomorphism, which proves the desired result.
\end{proof}
\begin{definition}[Flat Manifold] We say a Riemannian manifold is flat if it is isometric to a Euclidean space.
\end{definition}
\begin{theorem}['39, Myers-Steenrod]a If $M$ has finitely many components, then $\operatorname{Iso}(M, g)$ has a topology and smooth structure making it into a finite-dimensional Lie group acting smoothly on $M$.  
\end{theorem}
\subsection{Local Representations for Metrics}
If $(M, g)$ is a Riemannian manifold with local coordinates $(x^1, \dots, x^n)$ on open subset $U \subset M$, then we can write $g$ locally as 
$$g = g_{ij} dx^i \tensor dx^j,$$
for smooth functions $g_{ij}$ (note the usage of Einstein summation convention).
\pagebreak
\let \T \intercal
\section{Matrix Calculation}
 We have $B \in \R^{D \times D}, V \in \R^{D \times d \times d}, A \in \R^{d \times d \times D}$ and $ f(V) = \|VA - B\|_F^2$.
\begin{align*}
f(V + \delta V) &= \|B - (V + \delta V)A\|_F^2 \\
&\approx  f(V) + 2\<VA - B, \delta V A\> \\
&= f(V) + 2 \Tr((VA - B)^\T \delta V A)\\
&= f(V) + \Tr(2A(VA - B)^\T \delta V ) + o(\|\delta V \|)\\
\end{align*}
so $Df(V) = 2(VA - B) A^\T$.  We obtain from this that $VAA^\T = BA^\T$ giving $V = BA^\T(AA^\T)^{-1} \in \R^{D \times d \times d}$.

\pagebreak
\renewcommand{\listtheoremname}{List of Definitions and Theorems}
\listoftheorems[ignoreall,show={theorem,definition}]
\end{document}
