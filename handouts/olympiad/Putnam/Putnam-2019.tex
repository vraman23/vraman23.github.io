\documentclass[11pt]{scrartcl}
\usepackage[sexy]{evan}
\usepackage{graphicx}

\newcommand{\N}{\mathbb{N}}
\newcommand{\Z}{\mathbb{Z}}
\newcommand{\F}{\mathbb{F}}
\newcommand{\Q}{\mathbb{Q}}
\newcommand{\R}{\mathbb{R}}
\newcommand{\C}{\mathbb C}
\newcommand{\T}{\mathbb T}
\newcommand{\PP}{\mathbb P}
\newcommand{\supp}{\text{supp }}

\renewcommand{\Re}{\operatorname{Re}}
\renewcommand{\Im}{\operatorname{Im}}


\let \phi \varphi

%From Topology
\newcommand{\cT}{\mathcal{T}}
\newcommand{\cB}{\mathcal{B}}
\newcommand{\cC}{\mathcal{C}}
\newcommand{\cH}{\mathcal{H}}

\usepackage{answers}
\Newassociation{hint}{hintitem}{all-hints}
\renewcommand{\solutionextension}{out}
\renewenvironment{hintitem}[1]{\item[\bfseries #1.]}{}
\declaretheorem[style=thmbluebox,name={Problem}]{Prob}

\begin{document}
\title{2019 Putnam - Solutions}
\author{Vishal Raman}

\section{Problems}
\begin{Prob} Determine all possible values of the expression $$A^3 + B^3 + C^3 - 3ABC,$$
where $A, B, C$ are nonnegative integers.
\end{Prob}
\begin{proof}
Let $S = A^3 + B^3 + C^3 - 3ABC$.  We claim that $S$ attains all values such that $S \ne 3, 6 \pmod{9}$.

Note that the expression can be factored as 
$$A^3 + B^3 + C^3 - 3ABC = \left (\frac{A + B + C}{2}\right)\left((A-B)^2 + (B-C)^2 + (C-A)^2\right).$$

If $(A, B, C) = (A, A+1, A+2)$, then 
$$S = \frac{3A+3}{2}(1^2 + 1^2 + 2^2) = (3A + 3)(3) = 9A + 9,$$
so we can achieve all $S \equiv 0 \pmod 9.$

If $(A, B, C) = (A, A, A+1)$, then
$$S = \frac{3A+1}{2}(0^2+1^2+1^2) = 3A+1,$$
and if $(A, B, C) = (C+1, C+1, C)$, then 
$$S = \frac{3C+2}{2}(0^2+1^2+1^2) = 3C+2,$$
so we can achieve all $S \equiv 1, 2 \pmod{3}$.  

It suffices to show that if $S \equiv 0 \pmod{3}$, then $S \equiv 0 \pmod{9}$.  This implies that we cannot have $S \ne 3, 6 \pmod{9}$ as desired.  If $S \equiv 0 \pmod{3}$, then we must have $A+B+C \equiv 0 \pmod 3$ or $(A-B)^2 + (B-C)^2 + (C-A)^2 \equiv 0 \pmod 3$.  In the first case, then without loss of generality, we must have either $(A, B, C) \in \{(0, 0, 0), (1, 1, 1),( 2, 2, 2), (0, 1, 2)\}$.  In each of these cases, we can show that $(A-B)^2 + (B-C)^2 + (C-A)^2 \equiv 0 \pmod 3$.  Similarly, in the second case, we must have that $(A-B)^2 = (B-C)^2 = (C-A)^2 = 0, 1$.  In the first case $A = B = C$, which gives that $A+B+C \equiv 0 \pmod{3}$.  In the second case, the remainders of $A, B, C$ must be distinct mod $3$, which, without loss of generality, gives $(A, B, C) = (0, 1, 2)$ which implies that $A+B+C \equiv 0 \pmod{3}$, as desired.  In all cases, we show that both terms in the product are $0 \pmod {3}$, which implies that the product is $0 \pmod {9}$.
\end{proof}
\pagebreak
\begin{Prob} In the triangle $ABC$, let $G$ be the centroid, and let $I$
be the center of the inscribed circle. Let $\alpha$ and $\beta$ be
the angles at the vertices $A$ and $B$, respectively. Suppose that the segment $IG$ is parallel to $AB$ and that
$\beta = 2\arctan(1/3)$. Find $\alpha$.
\end{Prob}
\begin{proof}
We use complex numbers.  Let $B = 0$.  Then $\text{arg}(I) = \beta/2 = \arctan(1/3)$, so $I = k(3+i)$ for some $k \in \R^+$.  Without loss of generality, let $k = 1$.  Let $A = a$.  Then, $IG$ is parallel to $AB$ which implies that $\operatorname{Im}(B-A) = \Im(I-G)$.  Then $\Im(B-A) = 0$, so $\Im(I) = \Im(G) = 1$.  

Then, note that $\arg(I^2) = \arg(C)$, so $C = \ell(3+i)^2 = \ell(8+6i)$ for some $\ell \in \R^+$.  Then $G = \frac{A+B+C}{3} = \frac{A+C}{3}$, so $$1 = \Im(G) = \Im((A+C)/3) = \Im(C/3),$$ which implies that $\ell = \frac{1}{2}$.  Thus, $C = 4+3i$.  

Finally, $$I = \frac{|CB|A + |AC|B + |AB|C}{|AB|+|BC|+|CA|} = \frac{5a + a(4+3i)}{5+a+\sqrt{(4-a)^2+9}} = 3+i.$$

Hence,
$$5+a+\sqrt{(4-a)^2+9} = 3a,$$
which has solutions $a = 0, a = 4$.  Taking the positive solution, we have $A = 4$.  Then, note that $ABC$ is a right triangle with right angle at $A$, so $\alpha = \frac{\pi}{2}$.
\end{proof}
\pagebreak
\begin{Prob} Given real numbers $b_0, b_1, \dots, b_{2019}$ with $b_{2019} \ne 0$, let $z_1, z_2, \dots, z_{2019}$ be the roots in the complex plane of the polynomial $$P(z) = \sum_{k=0}^\infty b_kz^k.$$
Let $\mu = \frac{1}{2019}\sum_{k=1}^{2019}|z_k|$.  Determine the largest constant $M$ such that $\mu \ge M$ for all choices of $b_0, b_1, \dots, b_{2019}$ satisfying 
$$1 \le b_0 < b_1 < b_2 < \dots < b_{2019} \le 2019.$$
\end{Prob}
\begin{proof}
By the AM-GM inequality,
$$\mu = \frac{\sum_{k=1}^{2019}|z_k|}{2019} = \left (\prod_{k=1}^{2019} |z_k|\right )^{1/2019} = \left |\frac{b_0}{b_{2019}} \right|^{1/2019} \le (2019)^{-1/2019}.$$

We show that $M = (2019)^{-1/2019}$.  Let $\zeta = e^{\frac{2\pi i}{2020}}$ and let $z_i = M\zeta^i$. Notice that $|z_i| = M$ for each $i$ and the roots $z_1, z_2, \dots, z_{2019}$ satisfy the polynomial
$$0 = \frac{(z_i/M)^{2020} - 1}{(z_i/M) - 1} = M^{-2019}\left (\frac{z_i^{2020} - M^{2020}}{z_i - M}\right ) =\sum_{k=0}^{2019}z_i^{k}M^{-k}.$$

Hence, the polynomial $$P(z) = \sum_{k=1}^{2019}z_i^k2019^{k/2019}$$
satisfies the equality case $\mu = M$.  Furthermore, note that 
$b_0 = 1$, $b_{2019} = 2019$ and and $2019^{i/2019} < 2019^{j/2019}$ for all $i < j$.  Hence, $P$ satisfies the conditions.
\end{proof}
\begin{Prob}Let $f$ be a continuous real-valued function on $\R^3$.  Suppose that for every sphere $S$ of radius $1$, the integral of $f(x, y, z)$ over the surface of $S$ equals $0$.  Must $f(x, y, z)$ be identically $0$?
\end{Prob}
\begin{proof}
No.  Take $f(x, y, z) = \sin(\pi z)$.  Then,
\begin{align*}
\int_{\partial S} \sin(\pi z) dA &= \int_{0}^{2\pi} \int_0^\pi \sin(\pi \cos \theta) \sin \
\end{align*}
\end{proof}
\end{document}
